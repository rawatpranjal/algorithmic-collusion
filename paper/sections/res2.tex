\section{Results: Experiment 2}

This experiment tries to generalize the previous one to affliated values. The summary statistics of the data reveal no statistically significant correlation between \texttt{auction\_type\_code} and the main covariates (e.g., $\alpha$, $\gamma$, initialization, etc.); indicating that randomization went well. Unlike Experiment~1, the true valuations here are not fixed at 1, so some non-zero seller regret arises naturally. And we should not expect avg. revenues to ideally be 1. Nonetheless, our primary question remains: \emph{how does first-price compare to second-price} in terms of final revenue, regret, and other outcomes?

\paragraph{First price auctions reduce revenues.}
Table~\ref{tab:ate_exp2} summarizes the average treatment effects (ATEs) estimated via DoubleMLIRM. First-price auctions reduce final revenue by about $0.1185$ and increase seller regret by about $0.0627$, both at $p<0.001$. They also significantly raise price volatility ($+0.0360$) and produce a small but statistically significant uptick in winner entropy ($+0.0022$, $p\approx0.0415$). Neither time-to-converge nor no-sale rate changes meaningfully under first-price ($p>0.15$).

\begin{table}[H]
\centering
\caption{Final ATE Estimates (DoubleMLIRM) for All Outcomes in Experiment 2}
\label{tab:ate_exp2}
\begin{tabular}{lrrrr}
\toprule
\textbf{Outcome} & \textbf{ATE} & \textbf{StdErr} & \textbf{p-value} \\
\midrule
avg\_rev\_last\_1000    & -0.1185 & 0.0078 & 0.0000 \\
time\_to\_converge      &  0.0392 & 0.0274 & 0.1529 \\
avg\_regret\_of\_seller &  0.0627 & 0.0026 & 0.0000 \\
no\_sale\_rate          &  0.0022 & 0.0020 & 0.2863 \\
price\_volatility       &  0.0360 & 0.0023 & 0.0000 \\
winner\_entropy         &  0.0022 & 0.0011 & 0.0415 \\
\bottomrule
\end{tabular}
\end{table}

\paragraph{Private or common values do not matter.}
Despite \(\eta\) spanning the entire range \([0,1]\), our regression results show no statistically significant impact on any primary outcome in Experiment~2. The CATE BLP estimates for \(\eta\) and its squared term (\(\eta^2\)) are consistently insignificant across revenue, regret, volatility, and winner entropy, all with \(p\)-values greater than 0.1. Hence, although we modeled valuations via a linear affiliation process, these data suggest that \(\eta\) does not materially alter the relative performance of first- and second-price auctions in this setting.

\paragraph{More Bidders dampen the negative effects.}
Conditional average treatment effect (CATE) plots indicate that having more bidders moderates the revenue penalty from first-price auctions, aligning with economic intuition that heightened competition partially offsets strategic underbidding. Meanwhile, the discount factor $\gamma$ has no robust effect on these outcome gaps in our data, suggesting that future-oriented learning does not substantially alter the discrepancy between first- and second-price formats.

\begin{figure}[H]
\centering
\includegraphics[width=0.45\textwidth]{figures/e2_reg_bidder.png}
\hfill
\includegraphics[width=0.45\textwidth]{figures/e2_rev_bidder.png}
\caption{Illustrative partial-dependence (CATE) plots}
\label{fig:cate_bidders_exp2}
\end{figure}

\paragraph{Exploration and Initialization matter.}
We again find that exploration mode (\texttt{exploration\_code}) significantly shapes the revenue and regret gaps between first- and second-price. Table~\ref{tab:significant_gates} consolidates the statistically significant subgroup (GATE) differences across multiple outcomes.

\begin{table}[H]
\centering
\caption{Statistically Significant GATE Results for Experiment 2 (only $p<0.05$ shown).}
\label{tab:significant_gates}
\begin{tabular}{llllr}
\toprule
\textbf{Outcome} & \textbf{Binary Covariate} & \textbf{Diff} & \textbf{t} & \textbf{p} \\
\midrule
\texttt{avg\_rev\_last\_1000}    & exploration\_code &  0.1918 &  13.8902 & 0.0000 \\
\texttt{avg\_regret\_of\_seller} & exploration\_code & -0.0705 & -15.8271 & 0.0000 \\
\texttt{price\_volatility}       & init\_code        &  0.0093 &   2.0293 & 0.0424 \\
\texttt{price\_volatility}       & exploration\_code & -0.0590 & -14.7041 & 0.0000 \\
\texttt{winner\_entropy}         & exploration\_code & -0.0054 &  -2.3600 & 0.0183 \\
\bottomrule
\end{tabular}
\end{table}

As in Experiment~1, Boltzmann exploration dramatically reduces the first-price shortfall, while $\varepsilon$-greedy agents intensify these negative effects. We also observe a new (albeit smaller) effect of initialization (\texttt{init\_code}) on price volatility: starting Q-values at zeros (versus random) slightly amplifies the first-price volatility gap ($p\approx0.042$).

\paragraph{Conclusion.}
Even under affiliated valuations, first-price auctions yield significantly lower revenue, greater seller regret, and higher price volatility than second-price. Increasing the number of bidders alleviates some of these drawbacks, but does not fully eliminate them. Moreover, the choice of exploration policy remains pivotal: Boltzmann exploration consistently mitigates first-price losses, while $\varepsilon$-greedy exacerbates them. Notably, in this case we see a worsening of price volatility under first price auctions. From a practical standpoint, these results reinforce the conclusion that first-price auctions perform poorly once algorithms begin to learn.
