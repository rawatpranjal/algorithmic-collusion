\section{Experiment II: Q-Learning under Affiliated Valuations}

\begin{figure}[H]
  \centering
  \includegraphics[width=\textwidth]{figures/e2_trace}
  \caption{Representative learning trajectory for a single trial of Experiment~2 (first-price auction, $\eta = 0.5$, 2~bidders, 10{,}000 episodes). Top: rolling-mean revenue converging near the BNE prediction. Bottom: raw per-episode revenue in the final 2{,}000 episodes, illustrating residual variance around the equilibrium level.}
  \label{fig:e2_trace}
\end{figure}

Experiment~2 generalises the Q-learning framework to affiliated valuations by introducing the affiliation parameter $\eta \in \{0, 1\}$. The design is a $2^{11-1}$ Resolution~V half-fraction with 11 factors and 2{,}048 observations. The primary question is whether valuation interdependence, ranging from purely private values ($\eta = 0$) to near-common values ($\eta = 1$), alters the relative performance of first-price and second-price auctions under algorithmic bidding.

\subsection{Revenue and Seller Regret}

First-price auctions again significantly reduce revenue and increase seller regret. Table~\ref{tab:exp2_ranked_rev} confirms that auction type dominates the effect hierarchy, followed by exploration strategy and the number of bidders, replicating the factor ordering from Experiment~1. The main effects plots (Figures~\ref{fig:e2_main_rev} and~\ref{fig:e2_main_reg}) show that first-price auctions ($+1$) consistently underperform second-price auctions ($-1$) across both revenue and regret.

\input{tables/exp2_ranked_rev}

\begin{figure}[H]
  \centering
  \includegraphics[width=0.7\textwidth]{figures/e2_main_rev}
  \caption{Experiment~2: Main effects plot for average revenue. First-price auctions reduce revenue across all factor configurations.}
  \label{fig:e2_main_rev}
\end{figure}

\input{tables/exp2_ranked_reg}

\begin{figure}[H]
  \centering
  \includegraphics[width=0.7\textwidth]{figures/e2_main_reg}
  \caption{Experiment~2: Main effects plot for seller regret.}
  \label{fig:e2_main_reg}
\end{figure}

The affiliation parameter $\eta$ shows no statistically significant effect on any primary outcome. Despite spanning the full range from independent private values to near-common values, $\eta$ does not materially alter the relative performance of auction formats. This null result is striking: whether valuations are purely private or highly correlated, learning agents display similar bidding patterns, and the auction mechanism ($\text{first-price}$ vs.\ $\text{second-price}$) overshadows valuation interdependence in shaping long-run outcomes.

Exploration strategy remains pivotal. As in Experiment~1, Boltzmann exploration dramatically reduces the first-price shortfall, while $\varepsilon$-greedy agents intensify the negative effects. The number of bidders continues to moderate the auction type effect, with more bidders dampening the revenue penalty from first-price auctions. The interaction plot (Figure~\ref{fig:e2_int_rev}) displays these moderating relationships, showing how auction format interacts with exploration and competitive pressure.

\begin{figure}[H]
  \centering
  \includegraphics[width=0.7\textwidth]{figures/e2_int_rev}
  \caption{Experiment~2: Interaction plot for average revenue. The auction type $\times$ exploration and auction type $\times$ number of bidders interactions are among the strongest two-way effects.}
  \label{fig:e2_int_rev}
\end{figure}

\subsection{Price Volatility}

Unlike Experiment~1, where auction type had minimal effect on volatility, first-price auctions now significantly raise price volatility under affiliated valuations. Table~\ref{tab:exp2_ranked_vol} shows auction type among the significant effects for volatility. The main effects plot (Figure~\ref{fig:e2_main_vol}) confirms the directional increase. Initialisation strategy also contributes: starting Q-values at zeros rather than random values slightly amplifies the first-price volatility gap.

\input{tables/exp2_ranked_vol}

\begin{figure}[H]
  \centering
  \includegraphics[width=0.7\textwidth]{figures/e2_main_vol}
  \caption{Experiment~2: Main effects plot for price volatility.}
  \label{fig:e2_main_vol}
\end{figure}

\subsection{Model Fit and Significant Effects}

\input{tables/exp2_model_fit}

\input{tables/exp2_significant}

\subsection{Summary}

Even under affiliated valuations, first-price auctions yield significantly lower revenue, greater seller regret, and higher price volatility than second-price auctions. The affiliation parameter $\eta$ has no significant impact on any primary outcome, implying that the private-versus-common value distinction is irrelevant to the collusion dynamics studied here. Increasing the number of bidders alleviates some first-price drawbacks but does not eliminate them, and exploration strategy remains the key moderator. These results reinforce the conclusion from Experiment~1 that first-price auctions perform poorly under algorithmic bidding, and demonstrate that this finding is robust to the introduction of valuation interdependence.
