\section{Experimental Design}

We design three experiments to study repeated sealed-bid auctions under varying valuations and learning strategies. In all experiments, $n$ bidders submit discrete bids $b_i\in[0,1]$ each round. A reserve price $r\ge 0$ may invalidate low bids below $r$. If no valid bid meets $r$, no sale occurs and the revenue is zero; otherwise, the highest valid bid wins, and any tie among top bids is broken at random. We compare first-price (winner pays her own bid) and second-price (winner pays the second-highest bid) formats. When multiple bids tie for the highest valid bid, our implementation selects a single winner uniformly at random among those tied. Each experiment allows multiple parameter configurations, as described in corresponding tables, and collects the outcome metrics listed below.

\subsection{Experiment I: Constant Valuations} Here each bidder has a constant valuation $v_i=1.0$. This isolates how standard Q-learning responds to first- vs.\ second-price incentives when all players share the same private value of 1.0. Bidders maintain $Q$-values for pairs $(s,a)$, where $s$ can encode none, some, or all of the following: (i) the median of other bidders' previous-round bids and (ii) the previous winning bid. They update these $Q$-values either asynchronously (only the chosen action is updated per step) or synchronously (all actions are updated using counterfactual rewards). Table~\ref{tab:exp1params} summarizes the main parameters, including the learning rate $\alpha$, discount factor $\gamma$, exploration mode ($\varepsilon$-greedy or Boltzmann), number of bidders, reserve prices, and total training episodes. The combination of these factors tests how quickly bidders learn to match or shade their bids under the two payment rules.

\begin{table}[h]
\centering
\caption{Parameter settings for Experiment~1 (Constant Valuations).}
\label{tab:exp1params}
\begin{tabular}{l l}
\toprule
\textbf{Parameter} & \textbf{Possible Values}\\
\midrule
Number of bidders ($n$) & $\{2,4,6\}$\\
Reserve price ($r$) & $\{0.0,0.1,0.2,0.3,0.5\}$\\
Learning rate ($\alpha$) & e.g.\ $\{0.001,0.005,0.01,0.05,0.1\}$\\
Discount factor ($\gamma$) & e.g.\ $\{0.0,0.5,0.9,0.99\}$\\
Exploration & $\varepsilon$-greedy or Boltzmann\\
Q-update mode & Asynchronous or synchronous\\
State features & None / median-of-others / previous-winner\\
Bid grid resolution  & $\{11, 21\}$ discrete actions\\
Epsilon decay schedule & Linear or exponential\\
Number of episodes & e.g.\ $10^4$ or $10^5$\\
\bottomrule
\end{tabular}
\end{table}

\subsection{Experiment II: Affiliated Valuations with Q-Learning} Here bidders instead receive private signals $s_i\in[0,1]$, and their valuations become interdependent through an affiliation parameter $\eta\in[0,1]$ via $v_i=(1-0.5\,\eta)\,s_i + 0.5\,\eta\,\bigl(\tfrac{1}{n-1}\sum_{j\neq i}s_j\bigr)$. This lets $\eta=0$ represent purely private values and $\eta=1$ represent strong common-value elements, since each bidder’s valuation then weighs her own signal equally with that of her opponents. We again apply Q-learning with updates similar to Experiment\,1 but allow the current signal $s_i$ to appear in the bidder’s state. Table~\ref{tab:exp2params} highlights the key parameters, which still include choices for $\alpha$, $\gamma$, exploration strategy, number of bidders, reserve prices, and total episodes. By contrasting outcomes at different $\eta$ values, we observe how stronger affiliation (and thus more common-value features) affects revenue and convergence under first- vs.\ second-price rules.

\begin{table}[h]
\centering
\caption{Parameter settings for Experiment~2 (Affiliated Valuations + Q-learning).}
\label{tab:exp2params}
\begin{tabular}{l l}
\toprule
\textbf{Parameter} & \textbf{Possible Values}\\
\midrule
Number of bidders ($n$) & $\{2,4,6\}$\\
Reserve price ($r$) & $\{0.0,0.1,0.2,0.3\}$\\
Affiliation ($\eta$) & e.g.\ $\{0.0,0.25,0.5,0.75,1.0\}$\\
Learning rate ($\alpha$) & e.g.\ $\{0.001,0.005,0.01,0.05,0.1\}$\\
Discount factor ($\gamma$) & e.g.\ $\{0.0,0.5,0.9,0.99\}$\\
Exploration & $\varepsilon$-greedy or Boltzmann\\
Q-update mode & Asynchronous or synchronous\\
State features & e.g.\ $s_i,\,\text{median-of-others},\,\text{winner-bid}$\\
Number of episodes & e.g.\ $10^5$\\
\bottomrule
\end{tabular}
\end{table}

\subsection{Experiment III: Affiliated Valuations with Bandits} Here we preserve the signal-based, affiliated valuation model from Experiment\,2 but abandon Q-learning in favor of bandit-based methods. Each bidder treats possible bids as arms in either a standard multi-armed bandit or a linear contextual bandit (LinUCB), where the context could include the current signal plus optional median-of-others and winner-bid features. Exploration is controlled through an optimism bonus parameter $c$ rather than a learning rate or discount factor. A regularization parameter $\lambda$ may also be set to stabilize the bandit's linear estimates. Table~\ref{tab:exp3params} summarizes these parameters. This design highlights how structured bandit approaches might converge more efficiently than Q-learning, especially when valuations fluctuate with signals.

\begin{table}[H]
\centering
\caption{Parameter settings for Experiment~3 (Affiliated Valuations + Bandits).}
\label{tab:exp3params}
\begin{tabular}{l l}
\toprule
\textbf{Parameter} & \textbf{Possible Values}\\
\midrule
Number of bidders ($n$) & $\{2,4,6\}$\\
Reserve price ($r$) & $\{0.0,0.1,0.2,0.3,0.4,0.5\}$\\
Affiliation ($\eta$) & e.g.\ $\{0.0,0.25,0.5,0.75,1.0\}$\\
Bandit type & UCB or LinUCB\\
Exploration parameter ($c$) & e.g.\ $\{0.01,\dots,2.0\}$\\
Regularization ($\lambda$) & e.g.\ $\{0.1,1.0,5.0\}$\\
State features & e.g.\ $s_i,\,\text{median-of-others},\,\text{winner-bid}$\\
Number of rounds & e.g.\ $10^5$\\
\bottomrule
\end{tabular}
\end{table}

\subsection{Experiment IV: Budget-Constrained Pacing} Building on the affiliated-valuation model of Experiments~2 and~3, Experiment~4 introduces explicit budget constraints on each bidder. Rather than unconstrained learning, each of $n$ bidders must allocate a finite budget $B$ across $T$ rounds. We study two pacing algorithms: \emph{multiplicative pacing} (Lagrangian dual ascent, following Balseiro and Gur 2019) and a \emph{PID controller} (see Section~\ref{sec:algorithms}). Both apply a multiplier $\lambda_t$ to the raw valuation bid, with a hard cap enforcing that cumulative spending never exceeds $B$. The design retains the same affiliation parameter $\eta$ and auction format (first vs.\ second price) as bridge factors to connect with prior experiments. Three new factors (budget tightness, pacing aggressiveness, and update frequency) govern how the pacing control law responds to observed spending patterns. Table~\ref{tab:exp4params} details the factor levels.

\begin{table}[H]
\centering
\caption{Parameter settings for Experiment~4 (Budget-Constrained Pacing).}
\label{tab:exp4params}
\begin{tabular}{l l l}
\toprule
\textbf{Factor} & \textbf{Low level} & \textbf{High level}\\
\midrule
Algorithm & Multiplicative & PID \\
Auction type & Second-price & First-price \\
Number of bidders ($n$) & 2 & 4 \\
Budget tightness & 0.25 (tight) & 0.75 (loose) \\
Affiliation ($\eta$) & 0.0 (private) & 1.0 (common) \\
Aggressiveness & 0.5 & 2.0 \\
Update frequency & 1 (every round) & 100 (batched) \\
Initial multiplier & 0.5 & 1.0 \\
Reserve price ($r$) & 0.0 & 0.3 \\
\bottomrule
\end{tabular}
\end{table}

\subsection{Parameter Ranges in Experiments}
Table~\ref{tab:params} summarizes the parameters used across all experiments, their descriptions, and the ranges explored. A checkmark indicates that the parameter is applicable in a given experiment.

\begin{table}[H]
\centering
\caption{Parameter Ranges and Their Usage Across Experiments}
\label{tab:params}
\small
\begin{tabular}{l l c c c c}
\toprule
\textbf{Name} & \textbf{Description} & \textbf{E1} & \textbf{E2} & \textbf{E3} & \textbf{E4}\\
\midrule
$\alpha$ & Q-learning rate & \checkmark & \checkmark & & \\
$\gamma$ & Discount factor & \checkmark & \checkmark & & \\
$\varepsilon$ & E-greedy exploration prob & \checkmark & \checkmark & & \\
Boltzmann & Softmax exploration & \checkmark & \checkmark & & \\
$c$ & Bandit exploration param &  &  & \checkmark & \\
$r$ & Reserve price & \checkmark & \checkmark & \checkmark & \checkmark \\
$n$ & Number of bidders & \checkmark & \checkmark & \checkmark & \checkmark \\
$\eta$ & Affiliation parameter &  & \checkmark & \checkmark & \checkmark \\
\textit{Episodes} & Total training rounds & \checkmark & \checkmark & \checkmark & \checkmark \\
\textit{Sync/Async} & Q-learning modes & \checkmark & \checkmark & & \\
\textit{Median/Winner} & State features & \checkmark & \checkmark & \checkmark & \\
\textit{Bid grid}   & Number of discrete bid levels  & \checkmark & & & \\
\textit{Decay type}  & Epsilon decay schedule         & \checkmark & & & \\
Budget tightness & Fraction of max feasible budget & & & & \checkmark \\
Algorithm & Pacing control law & & & & \checkmark \\
Aggressiveness & Step-size multiplier & & & & \checkmark \\
Update frequency & Rounds per multiplier update & & & & \checkmark \\
Initial multiplier & Starting $\lambda_0$ & & & & \checkmark \\
\bottomrule
\end{tabular}
\end{table}

\subsection{Outcome Metrics} Our outcome metrics are consistent across all three experiments. We measure the \emph{average revenue in later rounds}, typically by averaging the revenue over the final 1000 episodes, which reflects how well the auction performs after strategies have stabilized. We also determine the \emph{time to converge} as the earliest round after which the rolling-average revenue remains within $\pm 5\%$ of its final mean. We define \emph{seller regret} as $1 - (\text{observed revenue})$ per round to gauge how far earnings fall below the maximal payoff of 1. We compute a \emph{no-sale rate} by noting the fraction of rounds in which no valid bid exceeds the reserve. We track \emph{price volatility} by taking the sample standard deviation of winning bids in later rounds. Finally, we measure \emph{winner entropy} to see whether a particular bidder tends to dominate, calculating the Shannon entropy of the empirical distribution of winners. Table~\ref{tab:outcomes} provides a concise summary of these common metrics that allow systematic comparisons of convergence speed, bidding stability, and revenue performance across all experiments.

\begin{table}[h]
\centering
\caption{Outcome metrics used in all experiments.}
\label{tab:outcomes}
\begin{tabular}{l l}
\toprule
\textbf{Metric} & \textbf{Description}\\
\midrule
Average revenue (later rounds) & Mean revenue in the final 1000 rounds \\
Time to converge & Round at which revenue stays in a $\pm 5\%$ band \\
Seller regret & $1-\text{(realized revenue per round)}$\\
No-sale rate & Fraction of rounds with all bids below $r$\\
Price volatility & Standard deviation of winning bids in later rounds \\
Winner entropy & Shannon entropy of bidder identity distribution \\
\bottomrule
\end{tabular}
\end{table}
