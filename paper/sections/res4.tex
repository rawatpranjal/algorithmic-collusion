\section{Experiment IV: Autobidding with Regenerating Budgets}
\label{sec:exp4_results}

\begin{figure}[H]
  \centering
  \includegraphics[width=\textwidth]{figures/e4_trace}
  \caption{Representative learning trajectory for a single trial of Experiment~4 (first-price auction, value-maximizer objective, 2~bidders, 100 episodes $\times$ 100 rounds). Top: platform revenue per episode. Middle: dual variable trajectories for each agent within the final episode. Bottom: mean bid-to-value ratio per episode with the competitive benchmark.}
  \label{fig:e4_trace}
\end{figure}

Experiment~4 shifts from tabular learners to dual-based autobidding agents operating under regenerating budget constraints. Each of $N$ advertisers bids in $D = 100$ episodes of $T = 1{,}000$ auctions, with budgets resetting between episodes while dual variables warm-start, mimicking daily budget cycles in ad exchanges. Valuations are drawn from bidder-specific log-normal distributions, introducing persistent asymmetry. The $2^3$ factorial crosses auction format (first-price vs.\ second-price), bidder objective (value-maximizer vs.\ utility-maximizer), and market thickness ($N = 2$ vs.\ $N = 4$), replicated across 15 seeds per cell (120 runs total). This design is motivated by the Price of Anarchy literature: first-price auctions with value-maximizers are predicted to achieve PoA $= 1$, while second-price auctions may reach PoA $= 2$.

\subsection{Efficiency and Welfare}

The effective Price of Anarchy, measured as the ratio of the offline optimum to realised liquid welfare, varies substantially across factor combinations. Tables~\ref{tab:exp4_ranked_rev} and~\ref{tab:exp4_ranked_reg} report the ranked significant effects for platform revenue and effective PoA, respectively.

\input{tables/exp4_ranked_rev}

\begin{figure}[H]
  \centering
  \includegraphics[width=0.7\textwidth]{figures/e4_main_rev}
  \caption{Experiment~4: Main effects plot for platform revenue across the three factors.}
  \label{fig:e4_main_rev}
\end{figure}

Allocative efficiency, the fraction of rounds in which the highest-value bidder wins, provides a complementary view. Under value-maximizing objectives, agents bid proportional to value divided by the dual variable, preserving the value ranking in bids. Under utility-maximizing objectives, the additional $1 + \lambda$ denominator can distort bid rankings relative to valuations, reducing allocative efficiency.

\input{tables/exp4_ranked_reg}

\begin{figure}[H]
  \centering
  \includegraphics[width=0.7\textwidth]{figures/e4_main_reg}
  \caption{Experiment~4: Main effects plot for effective Price of Anarchy.}
  \label{fig:e4_main_reg}
\end{figure}

The observed effective PoA values permit direct comparison with theoretical predictions from the autobidding literature. The number of bidders is the dominant factor ($t = 25.0$, $p < 10^{-47}$), followed by the objective$\times$n\_bidders interaction ($t = -6.0$, $p < 10^{-7}$); auction format is statistically insignificant ($t = 1.05$, $p = 0.30$). In four-bidder value-maximizing markets, effective PoA reaches approximately 1.04 under both auction formats, indicating that realised welfare falls within 4\% of the offline optimum, consistent with the prediction of Aggarwal et al.\ (2019) that first-price pacing equilibria attain near-unit PoA. Four-bidder utility-maximizing cells achieve PoA of 0.95 (first-price) and 0.94 (second-price), also close to full efficiency. Two-bidder settings yield PoA between 0.82 and 0.85 across all configurations. All observed values satisfy the welfare $\geq \frac{1}{2}$ guarantee of Gaitonde et al.\ (2023) by a wide margin. The absence of a significant auction format effect on PoA contrasts sharply with the revenue results, where first-price auctions systematically underperform: bid suppression under first-price reduces payments to the auctioneer without proportionally distorting allocative outcomes.

Allocative efficiency provides a complementary view. Under utility-maximizing objectives, agents achieve near-perfect efficiency (0.83 to 0.99 across cells), because the utility-maximizing bid formula preserves relative value rankings. Under value-maximizing objectives with four bidders, efficiency drops to 0.61--0.63: bid-to-value ratios exceeding 1.5 cause frequent misallocation, as the highest bidder is not necessarily the highest-value bidder when agents bid multiples of their values. The objective$\times$n\_bidders interaction ($t = 9.4$, $p < 10^{-15}$) captures this pattern, with more bidders amplifying the efficiency cost of value-maximizing objectives.

\subsection{Revenue and Budget Dynamics}

Platform revenue reflects the total payments collected by the auctioneer. Auction format exerts a direct effect through the payment rule: first-price auctions charge the winning bid, while second-price auctions charge the second-highest bid. Budget utilisation, the mean ratio of spend to budget across bidders, captures how tightly agents consume their allocated budgets. High utilisation indicates that budgets are binding constraints, while low utilisation suggests that the competitive environment does not require full budget expenditure.

The interaction plot (Figure~\ref{fig:e4_int_rev}) reveals how factor combinations jointly shape revenue outcomes. Market thickness interacts with auction format because additional bidders intensify competition differently under the two payment rules.

\begin{figure}[H]
  \centering
  \includegraphics[width=0.7\textwidth]{figures/e4_int_rev}
  \caption{Experiment~4: Interaction plot for platform revenue.}
  \label{fig:e4_int_rev}
\end{figure}

\subsection{Bidding Behaviour and Collusion Indicators}

The bid-to-value ratio (mean $b/v$) measures the aggressiveness of bidding relative to true valuations. Under competitive equilibrium, value-maximizers in first-price auctions bid $(N-1)/N$ of their value, while second-price bidders should bid truthfully ($b/v = 1$). The bid suppression ratio compares observed bid-to-value against these competitive benchmarks: values below one indicate that agents have learned to suppress bids below the competitive level, a signature of tacit collusion via budget-pacing dynamics.

\input{tables/exp4_ranked_vol}

\begin{figure}[H]
  \centering
  \includegraphics[width=0.7\textwidth]{figures/e4_main_vol}
  \caption{Experiment~4: Main effects plot for bid-to-value ratio.}
  \label{fig:e4_main_vol}
\end{figure}

Cross-episode drift, measured as the slope of the bid-to-value ratio across post-burn-in episodes, tests whether agents progressively learn to suppress bids over repeated campaign cycles, as predicted by the complex non-equilibrium dynamics of Paes Leme et al.\ (2024). The factorial ANOVA reveals that drift is statistically significant but economically negligible: all cell-level drift estimates are on the order of $10^{-5}$, and the model explains only 23\% of variance ($R^2 = 0.23$). First-price auctions show slightly more negative drift ($t = -2.96$, $p = 0.004$), as do markets with more bidders ($t = -3.40$, $p < 0.001$), but the magnitudes are too small to constitute meaningful progressive collusion over 100 episodes. This near-zero drift contrasts with the temporal dynamics observed in Q-learning (Experiment~2), where convergence times span tens of thousands of episodes, and suggests that dual-based pacing settles rapidly into a stable equilibrium that does not deteriorate over repeated campaigns.

\subsection{Learning Dynamics}

The warm-start benefit, measured as the revenue improvement from episode~1 to episode~2, quantifies how much dual variable persistence accelerates convergence. The bidder objective is the dominant factor ($t = 14.3$, $p < 10^{-26}$): utility-maximizing agents gain 78--141 revenue units from warm-starting, while value-maximizing agents in two-bidder first-price settings gain essentially nothing (0.18), indicating near-instant convergence. The objective$\times$auction\_type interaction ($t = 6.5$, $p < 10^{-9}$) reflects that first-price auctions amplify the warm-start advantage under utility-maximizing objectives, as the more complex first-price bidding problem benefits from retained dual variable estimates. Inter-episode volatility (the coefficient of variation of revenue across post-burn-in episodes) is dominated by the objective factor ($t = 19.3$, $p < 10^{-37}$): utility-maximizing agents exhibit higher volatility as the additional denominator term in the bid formula amplifies small fluctuations in the dual variable.

\subsection{Cross-Experiment Comparison}

The autobidding framework in Experiment~4 produces qualitatively different collusion dynamics from the tabular Q-learning of Experiments~1 and~2 and the contextual bandits of Experiment~3. Budget constraints impose an exogenous form of bid discipline that interacts with the endogenous strategic learning studied in prior experiments. The episodic structure with warm-starting creates a multi-timescale learning process: within-episode dual convergence on the fast timescale, and cross-episode strategy refinement on the slow timescale.

\subsection{Model Adequacy and Robustness}
\label{sec:exp4_robustness}

\input{tables/exp4_model_fit}

\input{tables/exp4_adequacy}

The $2^3$ factorial structure of Experiment~4 produces the cleanest inference among all four experiments. Table~\ref{tab:exp4_adequacy} reports model adequacy for the three primary responses. Platform revenue ($R^2 = 0.83$) and effective Price of Anarchy ($R^2 = 0.87$) show small PRESS gaps (0.022 and 0.017), indicating strong generalisability. The bid-to-value ratio achieves $R^2 = 0.99$ with a negligible gap of 0.001. The lack-of-fit test is marginally non-significant for platform revenue ($p = 0.060$) and non-significant for effective PoA ($p = 0.43$), but highly significant for bid-to-value ($p < 0.0001$), suggesting minor model misspecification at the extremes. LightGBM cross-validated $R^2$ is substantially lower than OLS $R^2$ for all responses, with catastrophically negative values for budget utilisation and dual variable CV. This confirms that the $2^3$ factorial with a linear model is the correct specification: nonlinear methods overfit severely with only 8 design cells, adding noise rather than signal.

\input{tables/exp4_inference_robust}

Table~\ref{tab:exp4_inference} reports only 2 HC3 flips across 78 effects tested (2.6\%), indicating near-homogeneous error variance across the design space. Holm--Bonferroni correction retains 50 of 62 OLS-significant effects (81\%), and Benjamini--Hochberg retains 57 (92\%). This strong robustness reflects the clean factorial structure and high signal strength of the autobidding experiment. Quantile regression reveals moderate heterogeneity for platform revenue (Figure~\ref{fig:e4_quantile_rev}): the objective coefficient grows from $-546$ at the 10th percentile to $-953$ at the 90th percentile, indicating that the revenue penalty from utility-maximizing objectives is concentrated among high-revenue configurations. The number of bidders effect remains stable across quantiles ($988$--$1{,}169$), confirming that market thickness benefits revenue uniformly across the distribution.

\begin{figure}[H]
  \centering
  \includegraphics[width=0.7\textwidth]{figures/e4_quantile_rev}
  \caption{Experiment~4: Quantile regression coefficients for platform revenue. The objective effect (utility-maximizer vs.\ value-maximizer) grows substantially toward the upper tail.}
  \label{fig:e4_quantile_rev}
\end{figure}

\input{tables/exp4_significant}

\subsection{Summary}

Autobidding with regenerating budgets introduces budget-mediated bid suppression that is qualitatively distinct from the strategic collusion in Experiments~1 through~3. Auction format, bidder objective, and market thickness jointly determine the welfare-revenue trade-off. The Price of Anarchy framework provides a natural benchmark for evaluating efficiency losses, connecting the simulation results to the theoretical predictions of Balseiro and Gur (2019), Aggarwal et al.\ (2019), and Deng et al.\ (2021).
