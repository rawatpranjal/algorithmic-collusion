\section{Experiment IV: Autobidding with Regenerating Budgets}
\label{sec:exp4_results}

\begin{figure}[H]
  \centering
  \includegraphics[width=\textwidth]{figures/e4_trace}
  \caption{Representative learning trajectory for a single trial of Experiment~4 (first-price auction, value-maximizer objective, 2~bidders, 100 episodes $\times$ 100 rounds). Top: platform revenue per episode. Middle: dual variable trajectories for each agent within the final episode. Bottom: mean bid-to-value ratio per episode with the competitive benchmark.}
  \label{fig:e4_trace}
\end{figure}

Experiment~4 shifts from tabular learners to dual-based autobidding agents operating under regenerating budget constraints. Each of $N$ advertisers bids in $D = 100$ episodes of $T = 1{,}000$ auctions, with budgets resetting between episodes while dual variables warm-start, mimicking daily budget cycles in ad exchanges. Valuations are drawn from bidder-specific log-normal distributions, introducing persistent asymmetry. The $2^3$ factorial crosses auction format (first-price vs.\ second-price), bidder objective (value-maximizer vs.\ utility-maximizer), and market thickness ($N = 2$ vs.\ $N = 4$), replicated across 50 seeds per cell (400 runs total). This design is motivated by the Price of Anarchy literature: first-price auctions with value-maximizers are predicted to achieve PoA $= 1$, while second-price auctions may reach PoA $= 2$.

\subsection{Efficiency and Welfare}

The effective Price of Anarchy, measured as the ratio of the offline optimum to realised liquid welfare, varies substantially across factor combinations. Tables~\ref{tab:exp4_ranked_rev} and~\ref{tab:exp4_ranked_reg} report the ranked significant effects for platform revenue and effective PoA, respectively.

\input{tables/exp4_ranked_rev}

\begin{figure}[H]
  \centering
  \includegraphics[width=0.7\textwidth]{figures/e4_main_rev}
  \caption{Experiment~4: Main effects plot for platform revenue across the three factors.}
  \label{fig:e4_main_rev}
\end{figure}

Allocative efficiency, the fraction of rounds in which the highest-value bidder wins, provides a complementary view. Under value-maximizing objectives, agents bid proportional to value divided by the dual variable, preserving the value ranking in bids. Under utility-maximizing objectives, the additional $1 + \lambda$ denominator can distort bid rankings relative to valuations, reducing allocative efficiency.

\input{tables/exp4_ranked_reg}

\begin{figure}[H]
  \centering
  \includegraphics[width=0.7\textwidth]{figures/e4_main_reg}
  \caption{Experiment~4: Main effects plot for effective Price of Anarchy.}
  \label{fig:e4_main_reg}
\end{figure}

\subsection{Revenue and Budget Dynamics}

Platform revenue reflects the total payments collected by the auctioneer. Auction format exerts a direct effect through the payment rule: first-price auctions charge the winning bid, while second-price auctions charge the second-highest bid. Budget utilisation, the mean ratio of spend to budget across bidders, captures how tightly agents consume their allocated budgets. High utilisation indicates that budgets are binding constraints, while low utilisation suggests that the competitive environment does not require full budget expenditure.

The interaction plot (Figure~\ref{fig:e4_int_rev}) reveals how factor combinations jointly shape revenue outcomes. Market thickness interacts with auction format because additional bidders intensify competition differently under the two payment rules.

\begin{figure}[H]
  \centering
  \includegraphics[width=0.7\textwidth]{figures/e4_int_rev}
  \caption{Experiment~4: Interaction plot for platform revenue.}
  \label{fig:e4_int_rev}
\end{figure}

\subsection{Bidding Behaviour and Collusion Indicators}

The bid-to-value ratio (mean $b/v$) measures the aggressiveness of bidding relative to true valuations. Under competitive equilibrium, value-maximizers in first-price auctions bid $(N-1)/N$ of their value, while second-price bidders should bid truthfully ($b/v = 1$). The bid suppression ratio compares observed bid-to-value against these competitive benchmarks: values below one indicate that agents have learned to suppress bids below the competitive level, a signature of tacit collusion via budget-pacing dynamics.

\input{tables/exp4_ranked_vol}

\begin{figure}[H]
  \centering
  \includegraphics[width=0.7\textwidth]{figures/e4_main_vol}
  \caption{Experiment~4: Main effects plot for bid-to-value ratio.}
  \label{fig:e4_main_vol}
\end{figure}

Cross-episode drift, measured as the slope of the bid-to-value ratio across post-burn-in episodes, tests whether agents progressively learn to suppress bids over repeated interactions. A negative drift indicates increasing bid suppression over the course of the experiment, consistent with emergent collusive dynamics. The dual variable coefficient of variation captures within-episode stability: lower CV indicates more stable pacing, which is a prerequisite for consistent strategic behaviour.

\subsection{Learning Dynamics}

The warm-start benefit, measured as the revenue improvement from episode~1 to episode~2, quantifies how much dual variable persistence accelerates convergence. Inter-episode volatility (the coefficient of variation of revenue across post-burn-in episodes) captures the stability of learned behaviour: low volatility indicates that the pacing algorithm has converged to a stable equilibrium across the episodic structure.

\subsection{Cross-Experiment Comparison}

The autobidding framework in Experiment~4 produces qualitatively different collusion dynamics from the tabular Q-learning of Experiments~1 and~2 and the contextual bandits of Experiment~3. Budget constraints impose an exogenous form of bid discipline that interacts with the endogenous strategic learning studied in prior experiments. The episodic structure with warm-starting creates a multi-timescale learning process: within-episode dual convergence on the fast timescale, and cross-episode strategy refinement on the slow timescale.

\subsection{Model Fit and Significant Effects}

\input{tables/exp4_model_fit}

\input{tables/exp4_significant}

\subsection{Summary}

Autobidding with regenerating budgets introduces budget-mediated bid suppression that is qualitatively distinct from the strategic collusion in Experiments~1 through~3. Auction format, bidder objective, and market thickness jointly determine the welfare-revenue trade-off. The Price of Anarchy framework provides a natural benchmark for evaluating efficiency losses, connecting the simulation results to the theoretical predictions of Balseiro and Gur (2019), Aggarwal et al.\ (2019), and Deng et al.\ (2021).
