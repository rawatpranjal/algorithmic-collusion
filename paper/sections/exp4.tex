\section{Appendix: Experiment 4 Technical Details}
\label{sec:exp4_appendix}

\subsection{Setup and Design}

Experiment~4 studies autobidding with regenerating budgets. Each of $n \in \{2, 4\}$ advertisers participates in a sequence of $D = 100$ episodes, each comprising $T = 1{,}000$ single-item auctions. Between episodes, budgets regenerate to their initial level while dual variables persist (warm-starting), mimicking the daily budget cycle in real ad exchanges.

Valuations follow a log-normal model with bidder-specific asymmetry. Each bidder $i$ draws a mean $\mu_i \sim \mathrm{Uniform}(0.5, 1.5)$ once per seed. In each round $t$, the valuation $v_{it} \sim \mathrm{LogNormal}(\mu_i, \sigma)$ with $\sigma = 0.3$. The budget per bidder per episode is $B_i = 0.5 \cdot \mathbb{E}[v_{it}] \cdot T$, where $\mathbb{E}[v_{it}] = \exp(\mu_i + \sigma^2/2)$.

All bidders use a multiplicative dual pacing algorithm. In each round the agent computes a bid as a function of its current dual variable $\mu_t$:
\begin{align}
  b_{it} &= \min\!\Big(\frac{v_{it}}{\mu_t},\; B_i - S_{it}\Big) & &\text{(value-maximizer),} \label{eq:bid_vmax} \\
  b_{it} &= \min\!\Big(\frac{v_{it}}{1 + \mu_t},\; B_i - S_{it}\Big) & &\text{(utility-maximizer),} \label{eq:bid_umax}
\end{align}
where $S_{it}$ is the cumulative spend at round $t$. The dual update follows
\begin{equation}
  \mu_{t+1} = \mathrm{clip}\!\Big(\mu_t \cdot \exp\!\big(\alpha_p\,(p_t - \rho)\big),\; 10^{-4},\; 100\Big),
  \label{eq:dual_update_exp4}
\end{equation}
with step size $\alpha_p = 1/\sqrt{T}$ and target spend rate $\rho = B_i / T$.

\subsection{Factorial Design}

The experiment uses a $2^3 = 8$ cell full factorial with three factors (Table~\ref{tab:exp4params}). Each cell is replicated across 50 independent seeds, yielding 400 total runs. This design estimates all main effects and interactions without aliasing.

\begin{table}[H]
\centering
\caption{Experiment~4 factor definitions.}
\label{tab:exp4params}
\begin{tabular}{lll}
\toprule
\textbf{Factor} & \textbf{Low ($-1$)} & \textbf{High ($+1$)} \\
\midrule
Auction format   & Second-price  & First-price \\
Bidder objective & Value-maximizer & Utility-maximizer \\
Number of bidders & 2 & 4 \\
\bottomrule
\end{tabular}
\end{table}

\subsection{Response Variables}

The following metrics are computed per run by averaging over the 90 post-burn-in episodes ($d \geq 10$):

\begin{table}[H]
\centering
\small
\caption{Experiment~4 response variables.}
\label{tab:exp4_responses}
\begin{tabular}{ll}
\toprule
\textbf{Metric} & \textbf{Definition} \\
\midrule
Platform revenue       & Total payments per episode \\
Liquid welfare         & Sum of winner valuations per episode \\
Effective PoA          & Offline optimum / liquid welfare \\
Budget utilisation     & Mean spend/budget across bidders \\
Bid-to-value ratio     & Mean $b/v$ across all bids \\
Allocative efficiency  & Fraction of rounds won by highest-value bidder \\
Dual variable CV       & CV of dual in last 200 rounds \\
No-sale rate           & Fraction of rounds with no valid bids \\
Winner entropy         & Shannon entropy of winner distribution \\
Warm-start benefit     & Revenue improvement episode~2 vs.~episode~1 \\
Inter-episode volatility & CV of revenue across post-burn-in episodes \\
Bid suppression ratio  & Observed btv / competitive btv \\
Cross-episode drift    & Slope of btv across episodes \\
\bottomrule
\end{tabular}
\end{table}

\subsection{Estimation}

Analysis proceeds in two stages. Stage~1 applies the same factorial ANOVA engine used in Experiments~1 through~3 to the 400-row run-level data. Stage~2 fits a panel regression on the approximately 36{,}000 episode-level observations with seed fixed effects and cluster-robust standard errors at the seed level. The panel specification is
\begin{equation}
  Y_{scd} = \alpha + \beta_1 \cdot \mathrm{FPA} + \beta_2 \cdot \mathrm{UtilityMax} + \beta_3 \cdot N_4 + \text{interactions} + \gamma_s + \varepsilon_{scd},
\end{equation}
where $s$ indexes seeds, $c$ indexes cells, and $d$ indexes episodes.

\input{tables/exp4_model_fit}

\input{tables/exp4_significant}
