\section{Literature}

There are a few kinds of literature that are relevant. First, a theoretical literature on auctions that assumes rational actors. Second, an experimental literature replaces rational actors with humans or algorithms. Third, empirical evidence on algorithmic adoption and collusion. Fourth, legal literature on possible gaps in regulation. 

The first thread is theoretical literature on auctions (Milgrom and Weber 1982, Krishna 2009). The second price auction is superior to the first price auction when bidders only have a noisy signal about the actual value of the item. The first price auction suffers from the risk of the winner's curse, which leads to conservative bidding, and this reduces auction revenues. These results can break down in repeated games, and the Folk Theorem says that if bidders are patient enough, any average winning bid can be supported by a suitable and credible threat of punishment e.g., tit-for-tat or grim trigger. Tacit collusion in repeated auctions will take the form of symmetrically suppressed bids or asymmetric bid rotation. 

Oligopoly theory also offers some insight into the determinants of tacit collusion in repeated games. Ivaldi et al. (2002) show that in repeated games, the degree of tacit collusion rises with a higher discount rate, fewer market participants, symmetric conditions, higher entry barriers, a high frequency of interaction, greater transparency, and data availability. The last three factors are magnified with algorithms and are often cited as the reason for concern for algorithmic collusion. 

The second thread consists of experiments. The experimental evidence with humans shows a distinctive departure from theory (Kagel and Levin 2011). Bids are generally above Nash prediction in both the first and second price auctions. Experiments with noisy signals show a significant presence of the winner's curse, which even experienced bidders are unable to avoid (Levin et al., 1996). Sequential auctions of identical items often show declining prices (Keser et al., 1996, Neugebauer et al., 2007). 

Experiments with algorithms have highlighted different mechanisms through which tacit collusion can occur. Waltman and Kamyak (2008), Dopologov (2021) and Abada et al (2022) show how certain exploration strategies can lead to collusive outcomes. Asker et al (2021) highlight asynchronous vs synchronous learning. Calvano (2020) studies retaliatory strategies in simultaneous pricing games. Klein (2021) studies sequential pricing games and granularity of action spaces. Hansen et al (2021) study collusion arising from misspecified prediction and correlated price experimentation. Hettich (2021), Han (2022) and Zhang (2021) study deep reinforcement learning and compare different sampling strategies. Most of the literature on algorithmic collusion has focused on pricing and the only experiments with auctions can be found in Banchio and Skrzypacz (2022).

A few papers look at algorithms learning to bid in auctions. Bandyopadhyay et al. (2008), study reverse auctions with reinforcement learning and find that in simple cases mixed strategy equilibrium can be attained. Tellidou et al.,(2007) study electricity markets and find that tacit collusion is easy to sustain even under competitive conditions. Banchio and Skrzypacz (2022) study auction design with Q-learning bots and found that the first price auction can lead to collusive outcomes while the second price auction does not. Criticisms of simulation-based studies are that settings are too stylized, using competitive rather than monopolistic benchmarks is unrealistic, and algorithms tested are too unrealistic (Kühn \& Tadelis 2017, Schwalbe 2018).

The third thread looks at the pervasive adoption of algorithms and actual evidence for algorithmic collusion. Chen et al. (2016) studied 1,641 best-seller products on Amazon and detected that about 543 had adopted some form of algorithmic pricing. A 2017 OECD report titled ``Algorithms and Collusion" found that \textit{``Two-thirds of them [ecommerce firms] use automatic software programs that adjust their own prices based on the observed prices of competitors"}. A 2023 eMarketer report shows that algorithms are dominating bidding in display and sponsored search auctions across the globe. Brogaard et al., (2014) find that algorithms have come to dominate trading in the double auction markets. 

These developments have led to limited evidence of algorithmic collusion. Assad et al. (2020) study the adoption of algorithmic pricing in German gasoline markets. They find that adoption increases margins by 9\% in competitive markets and upto 28\% in duopolies. Brown and Mackay (2021) five largest online retailers in over-the-counter allergy medications. The use of automated software to change prices quickly leads to prices that are 10 percent higher than otherwise. Musoloff (2022) studied data from Amazon Marketplace and found that adopting algorithmic pricing reduces prices on average but leads to ``resetting" price cycles where participants regularly increase prices at night to cajole the others into following suit. We also see a few legal cases already. \footnote{The 2015 Topkins (US) and 2016 GB Posters (UK) cases brought to light price fixing via pricing algorithms on Amazon. In 2018, after the insolvency of Air Berlin, Lufthansa abused its temporary monopoly power by raising prices roughly by 25\% through pricing algorithms. In 2021, Google was fined 2.42 billion Euro for using algorithms to exclusively place its own Shopping services on Google search while demoting rival providers. Similarily, Amazon is currently defending a claim that its algorithms put its own products over rivals on its lucrative ‘Featured Offer’ placement. There have also been hub-and-spoke conspiracies. In 2016, e-Turas the Lithuanian travel website, was charged for imposing caps on discount rates proposed by travel agencies. In 2018, Accenture was accused of using its software Partneo to coordinate price increases on auto parts, dramatically increasing revenue for major carmakers.}

The fourth thread looks at possible gaps in regulation. Tacit algorithmic collusion is currently not prohibited by law. In the United States, anti-collusion laws require evidence of ``actionable agreement" over mere interdependent behavior. A report at the OECD Competition Committee \footnote{``Algorithms and Collusion" - A note submitted by the United States to the OECD Competition Committee.} reiterates that in the U.S. firms have the freedom to decide their own prices and implement any kind of pricing technology; only decisions taken with an understanding with other competitors is prosecutable. Similarily, European Law (Article 101 TFEU) outlaws three types of collusion: agreements, decisions, and concerted practices. Again, this does not include tacit collusion. 

There are many ways in which algorithms can participate in collusion (Ezrachi and Maurice 2017). First, they can act as messengers and conduits for human collusion and cartels e.g., price fixing on Amazon via algorithms. Here, an agreement to collude is established clearly. Second, they can be part of a hub-and-spoke conspiracy where the developers of the hub use a single algorithm to set prices for many spokes e.g., Uber's algorithm setting taxi rates. Agreement to collude is harder to establish in such vertical agreements, and so evidence for intent must be found. Third, humans can unilaterally set up to behave in predictable ways, and some of them lead to coordination with others e.g. gasoline pricing algorithms in Brown and MacKay (2021). Here agreement is nonexistent, and intent to collude must be established. Lastly, algorithms are given the objective to maximize profits and, through sophisticated learning and experimentation, begin to collude with other similar algorithms e.g. Deep Q-learning. In this case, neither agreement nor intent can be established, and the case would be difficult to prosecute. This paper addresses this last case. It is this last case of algorithmic tacit collusion that this paper considers. 

To summarize: (1) Second-price auctions are generally considered theoretically superior to first-price auctions, but in repeated auctions and complex learning dynamics this may not be true. In practise the first-price auction remains a popular choice for platforms. (2) There has been widespread adoption of algorithms in marketplaces but only a few cases of algorithmic collusion have been detected. There is little work dones on detecting algorithmic collusion. (3) Simulations and experiments have highlighted many mechanisms and conditions through which tacit algorithmic collusion is possible. Many of this has been done under controlled laboratory environments and not field experiments. (4) There seems to be a large gap in policy and regulation since the innovation in algorithms continues to outpace legal developments. 
