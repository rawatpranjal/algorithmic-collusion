\section{Experiment III: Contextual Bandits}

\begin{figure}[H]
  \centering
  \includegraphics[width=\textwidth]{figures/e3_trace}
  \caption{Representative learning trajectory for a single trial of Experiment~3 (LinUCB, first-price auction, $\eta = 0.5$, 2~bidders, 5{,}000 rounds). Top: rolling-mean revenue with BNE benchmark. Middle: bid versus signal in the final 1{,}000 rounds, overlaid with the BNE bid function. Bottom: cumulative average reward per bidder.}
  \label{fig:e3_trace}
\end{figure}

Experiment~3 replaces Q-learning with LinUCB contextual bandits, testing whether a more sophisticated exploration mechanism improves seller outcomes. The design is a $2^8$ full factorial with 8 factors and 512 observations. The outcome range widens substantially relative to Experiment~2: certain runs converge to near-zero revenue or universal no-sales, while others perform comparably to Q-learning. Mean revenue drops and mean seller regret rises, with more polarised extremes of volatility, winner entropy, and no-sale rates. This highlights how contextual exploration can magnify variability in auction outcomes rather than reliably benefiting the seller; a finding that challenges the common criticism of Q-learning studies, which argues that firms would adopt more efficient exploration algorithms to improve outcomes.

\subsection{Revenue and Seller Regret}

Auction type remains the dominant factor, with the largest absolute $t$-statistic ($|t| \approx 12$) for both revenue and regret. Table~\ref{tab:exp3_ranked_rev} shows that auction type is followed by number of bidders and the exploration parameter $c$. First-price auctions yield substantially lower final revenue and higher seller regret, as the main effects plots (Figures~\ref{fig:e3_main_rev} and~\ref{fig:e3_main_reg}) confirm.

\input{tables/exp3_ranked_rev}

\begin{figure}[H]
  \centering
  \includegraphics[width=0.7\textwidth]{figures/e3_main_rev}
  \caption{Experiment~3: Main effects plot for average revenue. First-price auctions produce markedly lower revenue than second-price.}
  \label{fig:e3_main_rev}
\end{figure}

\input{tables/exp3_ranked_reg}

\begin{figure}[H]
  \centering
  \includegraphics[width=0.7\textwidth]{figures/e3_main_reg}
  \caption{Experiment~3: Main effects plot for seller regret. The mirror image of the revenue pattern.}
  \label{fig:e3_main_reg}
\end{figure}

The affiliation parameter $\eta$ again shows no statistically significant effect on any primary outcome, consistent with Experiment~2. Even in this bandit-based setting, increasing affiliation does not meaningfully change the outcome gaps between auction formats.

Contrary to Q-learning experiments, more bidders \emph{worsen} first-price performance under LinUCB. Increased competition does not mitigate strategic underbidding; instead, it amplifies it, leading to worse revenue and regret as the number of bidders rises. This reversal from Experiments~1 and~2 suggests that optimism-based exploration interacts with competitive pressure differently than $\varepsilon$-greedy or Boltzmann exploration.

The exploration parameter $c$ has a significant negative effect on revenue, indicating that higher exploration incentives reduce final revenues. This challenges the assumption that efficient exploration benefits sellers: LinUCB's optimism-based mechanism may reinforce risk-averse bidding by inflating uncertainty estimates around aggressive actions that rarely win. The interaction plot (Figure~\ref{fig:e3_int_rev}) shows these moderating effects and the departures from additivity among the top factorial interactions.

\begin{figure}[H]
  \centering
  \includegraphics[width=0.7\textwidth]{figures/e3_int_rev}
  \caption{Experiment~3: Interaction plot for average revenue under LinUCB. Non-parallel lines indicate factor interdependencies, including the bidder-count reversal.}
  \label{fig:e3_int_rev}
\end{figure}

\subsection{Price Volatility}

Under LinUCB bandits, price volatility is \emph{lower} in first-price auctions, the opposite direction from Experiment~2. Table~\ref{tab:exp3_ranked_vol} and the main effects plot (Figure~\ref{fig:e3_main_vol}) confirm this reversal. The result suggests that first-price auctions under LinUCB produce more uniform (though lower) winning bids, as agents converge to stable but suboptimal bid-shading equilibria. Reserve prices also show a significant negative effect on volatility, stabilising winning bids by disqualifying extreme low bids.

\input{tables/exp3_ranked_vol}

\begin{figure}[H]
  \centering
  \includegraphics[width=0.7\textwidth]{figures/e3_main_vol}
  \caption{Experiment~3: Main effects plot for price volatility.}
  \label{fig:e3_main_vol}
\end{figure}

\subsection{Model Fit and Significant Effects}

\input{tables/exp3_model_fit}

\input{tables/exp3_significant}

\subsection{Summary}

Replacing Q-learning with LinUCB contextual bandits amplifies variability in auction outcomes and does not improve seller welfare. First-price auctions produce substantially lower revenue and higher regret under bandits than under Q-learning, with the revenue gap widening dramatically. The exploration parameter $c$ reduces rather than improves revenue, and more bidders worsen first-price performance; both reversals from Q-learning experiments. These findings demonstrate that algorithmic sophistication does not inherently mitigate the disadvantages of first-price auctions and can exacerbate negative outcomes when the exploration mechanism reinforces conservative bidding.
